% !TeX root = ../../01_git-vorgehensmodelle.tex

Die vorliegende Arbeit widmet sich verschiedenen Git\hyp Vorgehensmodellen, die für Softwareentwicklungsprojekte in größeren Teams in Betracht kommen. Ziel ist es, ein grundlegendes Verständnis für die Anwendung dieser Strategien zu vermitteln.

Im ersten Teil der Arbeit werden die ausgewählten Git\hyp Vorgehensmodelle im Einzelnen vorgestellt. Im Anschluss erfolgt eine Gegenüberstellung und Bewertung der verschiedenen Vorgehensmodelle anhand ausgewählter Kriterien. Dadurch sollen die Leser:innen in die Lage versetzt werden, die Stärken und Schwächen der verschiedenen Ansätze zu verstehen und entsprechende Entscheidungen für die Auswahl eines geeigneten Vorgehensmodells in Abhängigkeit von den Projektanforderungen zu treffen.

Die Arbeit kommt zur Erkenntnis, dass Gitflow und Trunk\hyp based Development als Basis\hyp Workflow für unterschiedliche Szenarien geeignet sind. Diese sollten jedoch an den entsprechenden Team- und Projektkontext angepasst werden. Für Open\hyp Source\hyp Projekte eignet sich zusätzlich der Forking\hyp Workflow als übergeordnetes Vorgehensmodell.
