% !TeX root = ../../../01_git-vorgehensmodelle.tex

\section{Zusammenfassung}
\label{sec:zusammenfassung}

Zusammenfassend konnten im Rahmen dieser Arbeit die häufig genutzten Git\hyp Workflows vorgestellt und unter ausgewählten Kriterien gegenübergestellt werden. 

Dabei wurden die Branching\hyp Strategien Feature\hyp Branching, Release\hyp Branching und Task\hyp Branching vorgestellt. Die Feature\hyp Branching\hyp Strategie isoliert die Entwicklung neuer Features auf eigene Branches und schafft so eine isolierte Arbeitsumgebung für die Entwicklung neuer Features. Die Release\hyp Branching\hyp Strategie verwendet dedizierte Branches für unterschiedliche Releases um eine gleichzeitige Unterstützung mehrerer Versionen zu ermöglichen. Die Task\hyp Branching\hyp Strategie, verknüpft Branches mit Tickets oder Issues aus Vorgangs\hyp Tracking\hyp Software.

Des Weiteren wurde das Vorgehensmodell der Trunk\hyp based Development vorgestellt welches auf langfristige Branches verzichtet und Änderungen direkt in den Hauptentwicklungszweig integriert. Unreife Funktionen können in diesem Modell durch Feature\hyp Flags isoliert werden.

Im Anschluss wurde der Gitflow\hyp Workflow beschrieben. Es basiert auf einem zentralen Repository (\texttt{origin}), das als \emph{Single\hyp Source\hyp of\hyp Truth} dient, und definiert verschiedene Arten von Branches für einen klaren Entwicklungsfluss.

Die Haupt\hyp Branches sind \texttt{main} und \texttt{develop}. Der \texttt{main}-Branch repräsentiert den Produktionszustand und wird für offizielle Releases verwendet, während \texttt{develop} ein Integrationsbranch ist, der die gesamte Versionshistorie enthält. Release\hyp Branches ermöglichen die Vorbereitung von Releases, wobei Bugfixes und Metadaten hinzugefügt werden können. Hotfix\hyp Branches beheben kritische Fehler in \texttt{main}, werden dort integriert und anschließend in \texttt{develop} zurückgeführt.

Zuletzt wurde der Forking\hyp Workflow vorgestellt. Dabei wird eine Kopie des Haupt\hyp Repositories erstellt, auf der Entwicklungen und Änderungen vorgenommen werden können. Anschließen können Entwickler:innen Änderungen in Form von Pull Requests beim Haupt\hyp Repository einreichen. Diese müssen von den Projektverwalter:innen angenommen werden, bevor sie in das Haupt\hyp Repository integriert werden.

Den Leser:innen wurde, durch die Vorstellung der häufig genutzten Vorgehensmodelle, eine Basis zur Entwicklung eines auf die spezifische Arbeitsumgebung zugeschnittenen Modells vermittelt. 

Die Herausarbeitung der Unterschiede im \autoref{sec:diskussion} ermöglicht es den Leser:innen, passende Entscheidungen bei der Auswahl eines geeigneten Git\hyp Workflows als Basis für ihre konkreten Arbeitsumgebungen zu treffen.

Zusammenfassend ergeben sich dabei folgende Erkenntnisse: Für kleinere, erfahrene Teams, die eine schnelle Veröffentlichung und Feedback anstreben, ist Trunk\hyp based Development zu empfehlen. Kleinere Teams mit unterschiedlichen Erfahrungsgraden können entweder auf Gitflow oder eine Kombination der in \autoref{sec:task:task-branch} vorgestellten Workflows zurückgreifen. Bei Open\hyp Source\hyp Projekten bietet sich der Forking\hyp Workflow als übergeordneter Ansatz an, der idealerweise mit einem passenden Workflow für die Maintainer kombiniert wird. Für alle anderen, insbesondere für große Teams, stellt Gitflow eine solide Basis dar.

Dabei wurde deutlich, dass kein Vorgehensmodell als universelle Lösung anwendbar ist, sondern dass die Wahl des geeigneten Modells stark von den spezifischen Anforderungen und Charakteristika des Softwareentwicklungsprojekts abhängt. Außerdem sollte ein Workflow an den entsprechenden Kontext angepasst werden, um den Anforderungen des Projektes und des Teams gerecht zu werden.
