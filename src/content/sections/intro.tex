% !TeX root = ../../../01_git-vorgehensmodelle.tex

\setcounter{page}{1}

\section{Einleitung}
\label{sec:einleitung}

\lipsum[1]

Dies ist ein Paragraph, damit die Literatur nicht leer ist~\cite{atlassianGitflowWorkflow}.



% ------------------------
% MOTIVATION
% ------------------------

\subsection{Motivation}
\label{sec:einleitung:motivation}

\hl{\textbf{hier Probleme sammeln}}

\begin{itemize}
    \item Entwickler:innen produzieren unterschiedlich hohe Codequalitäten $\to$ wie kann Codequalität dennoch insgesamt hoch gehalten werden?
    \item erschwerte Kommunikation, bspw. unterschiedliche Büros, Zeitzonen, Teams, Sprachen, Kulturen etc.
    \item schwierige Koordination von vielen Aufgaben durch eine große Zahl an Entwickler:innen
    \item stark wachsender Quellcode $\to$ Unübersichtlichkeit des Projekts
    \item Gewährleistung von Standards, Zertifizierungen etc. von kritischer Software
\end{itemize}



% ------------------------
% AUFBAU
% ------------------------

\subsection{Aufbau}
\label{sec:einleitung:aufbau}

\lipsum[3]
