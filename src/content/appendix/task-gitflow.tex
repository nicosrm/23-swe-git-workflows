% !TeX root = ../../../01_git-vorgehensmodelle.tex

\section{Aufgabe: Gitflow}
\label{sec:task:gitflow}

\begin{enumerate}
    \item Erstellen Sie ein neues Repository und alle langlebigen Branches, die für Gitflow benötigt werden.
    
    \item Sie bekommen ein Ticket, welches verlangt, dass ein Sortieralgorithmus als neue Funktion implementiert werden soll. Zum Test soll ein Array der Länge 100 sortiert werden. Erstellen Sie mind. einen entsprechenden Commit auf dem für diesen Workflow entsprechenden Branch.
    
    \item Integrieren Sie die Implementierung auf den oder die vorgesehenen Branches. Das Ticket ist für den nächsten Release angesetzt.
    
    \item Es steht der Release der ersten Version \verb|0.1| an. Verfahren Sie dem Workflow entsprechend und bereiten den Release vor.
    
    \item Es kommt ein weiteres Ticket hinzu: implementieren Sie ein Programm, welches \verb|Hello, World!| ausgibt. Das Ticket ist für den nächsten Release angesetzt. Verfahren Sie dem Workflow entsprechend.
    
    \item Stellen Sie sich vor, dass vorbereitende Arbeiten für den Release \verb|0.1| durchgeführt wurden. Der Release ist nun bereit. Verfahren Sie dem Workflow entsprechend.
    
    \item Der nächste Release steht an. Bereiten Sie den Release \verb|0.2| vor. Verfahren Sie entsprechend dem Workflow, sodass das Release\hyp Team die vorbereiteten Aufgaben durchführen kann.
    
    \item Das Release\hyp Team hat den Release freigegeben. Verfahren Sie entsprechend des Workflows.
    
    \item Es wurde ein kritischer Fehler im letzten Release gefunden! Es soll nicht \verb|Hello, World!| sondern \verb|Hello, WORLD!| ausgegeben werden. Beheben Sie diesen kritischen Fehler in der \emph{Production}-Version. Die neue Version soll \verb|0.2.1| sein. Verfahren Sie entsprechend des Workflows.
\end{enumerate}
