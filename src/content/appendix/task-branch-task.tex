% !TeX root = ../../../01_git-vorgehensmodelle.tex

\section{Aufgabe: Branching Strategien}
\label{sec:task:task-branch}

\begin{enumerate}
    \item Erstellen Sie ein neues Repository.

    \item Erstellen Sie ein \enquote{Hello World}-Programm in einer Programmiersprache Ihrer Wahl.

    \item Fügen die dieses Programm als initialen Commit Ihrem Repository hinzu.

\end{enumerate}

\subsection{Feature\hyp Branching}

\begin{enumerate}
    \item Erstellen Sie einen neuen Feature\hyp Branch.

    \item Wechseln Sie auf den neuen Feature\hyp Branch und bauen Sie hier einen Fehler in Ihr Programm ein.
    
    \item Wechseln Sie auf den Haupt\hyp Branch und testen Sie ob ihr Programm noch funktionsfähig ist.
    
    \item Beheben Sie den Fehler auf dem Feature\hyp Branch und ändern Sie \enquote{Hello World} zu \enquote{Hallo Welt}.
    
    \item Erstellen Sie eine Pull Request.
    
    \item Bestätigen Sie Ihre Pull Request und testen Sie ob die Änderung nun auf dem Haupt\hyp Branch übertragen wurde.
\end{enumerate}

\subsection{Release\hyp Branching}

\begin{enumerate}
    \item Erstellen Sie einen neuen Release\hyp Branch.
 
    \item Wechseln Sie auf den neuen Release\hyp Branch und fügen Sie zusätzlich zu \enquote{Hallo Welt} die Textausgabe \enquote{Release 1} hinzu.
    
    \item Wechseln Sie auf den Haupt\hyp Branch und testen Sie das Programm. Die Textausgabe \enquote{Release 1} sollte nicht mehr erscheinen.
    
    \item Wechseln Sie erneut auf den Release\hyp Branch und ändern Sie \enquote{Hallo Welt} zu \enquote{Hallo Welt!}.

    \item Fügen Sie die Änderung durch einen Merge oder durch Cherry\hyp Pick der gewünschten Commits dem Haupt\hyp Branch hinzu.
    
    \item Testen Sie auf dem Haupt\hyp Branch ob Ihre Änderungen wirksam sind.
\end{enumerate}

\subsection{Task\hyp Branching}

\begin{enumerate}
    \item Erstellen Sie ein neues Ticket / Issue in welchem beschrieben wird, dass dem \enquote{Hallo Welt!} zwei weitere Ausrufezeichen angehängt werden sollten.

    \item Erstellen Sie einen neuen Branch und benennen ihn entsprechend.
    
    \item Prüfen Sie ob der Git\hyp Host\hyp Service Ihrer Wahl eine Verknüpfung zwischen dem Ticket / Issue und dem Branch hergestellt hat.
\end{enumerate}

\subsection{Kontrollfragen}

Wechseln Sie in den Haupt\hyp Branch und lassen Sie sich den Git Log als Graph anzeigen. Nutzen Sie dafür den Befehl \texttt {git log \textendash\textendash graph}.

\begin{itemize}
    \item Wurde der Feature\hyp Branch korrekt vom Haupt\hyp Branch abgespalten und nach den Änderungen wieder mit diesem vereint?
    
    \item Wurden die gewünschten Änderungen auf dem Release\hyp Branch richtig in den Haupt\hyp Branch übernommen?
    
    \item Besteht der Release\hyp Branch nach Abschluss der Aufgabe weiterhin oder wurde dieser durch einen Merge in den Haupt\hyp Branch gelöscht? Überprüfen Sie anhand von \autoref{sec:workflows:branching}.
    
    \item Wurde der Task\hyp Branch korrekt vom Haupt\hyp Branch abgespalten und nach den Änderungen wieder mit diesem vereint?
\end{itemize}
