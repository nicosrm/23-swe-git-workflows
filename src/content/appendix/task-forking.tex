% !TeX root = ../../../01_git-vorgehensmodelle.tex

\section{Aufgabe: Forking\hyp Workflow}
\label{sec:task:forking}


\begin{enumerate}
    \item Besuchen Sie das Test\hyp Repository unter folgender URL: \url{https://gitlab.dit.htwk-leipzig.de/23-swe-thema-1/fork-me}
    
    \item Erstellen Sie einen Fork des oben genannten Repositories auf Ihrem eigenen GitLab\hyp Profil.
  
    \item Schreiben Sie ein \enquote{Hallo, Welt!} Programm in einer Programmiersprache Ihrer Wahl.
  
    \item Implementieren Sie das Programm in Ihrem Fork. Nutzen Sie dabei ggf. das Gitflow\hyp Vorgehensmodell.
  
    \item Erstellen Sie eine Pull Request, um Ihre Änderungen in das Test\hyp Repository zu integrieren. Achten Sie dabei darauf, als Ziel\hyp Branch den entsprechenden Branch im Test\hyp Repository anstatt den eigenen Fork auszuwählen.
  
\end{enumerate}

Kontrollfragen:

\begin{itemize}
  \item Wird in ihrem Fork das Test\hyp Repository entsprechend als Upstream\hyp Repository angezeigt, beispielsweise \enquote{Forked from Beleg 01 - Git-Vorgehensmodelle / Fork me}?

  \item Wird die Pull Request im Test\hyp Repository angezeigt? 
  
  \item Enthält die Pull Request die neu hinzugefügten Änderungen?
\end{itemize}
