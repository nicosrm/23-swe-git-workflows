% !TeX root = ../../../01_git-vorgehensmodelle.tex

\section{Aufgabe: Trunk-based Development}
\label{sec:task:tbd}

\begin{enumerate}
    \item Erstellen Sie ein neues Repository. Initialisieren Sie den langlebigen Branch.
    \item Sie erhalten ein Ticket zur Implementierung des Bubblesort\hyp Algorithmus. Implementieren Sie die Änderungen direkt im Trunk.
    \item Sie erhalten ein Ticket zur Implementierung des Insertionsort\hyp Algorithmus. Erstellen Sie einen kurzlebigen Branch, implementieren Sie den Algorithmus und integrieren Sie ihn direkt in den Trunk.
    \item Implementieren Sie ein experimentelles Feature (zum Beispiel eine alternative Sortiermethode) und integrieren Sie es mithilfe von Feature Flags in den Trunk.
    \item Stellen Sie sich vor, der Trunk ist bereit für einen Release (1.0).
    Setzen Sie eine Versionsnummer (1.0) direkt im Trunk.
    Führen Sie den Release direkt aus dem Trunk durch.
    \item Sie erhalten das Ticket Ihre alternative Sortiermethode auszuarbeiten. Verfahren Sie entsprechend des TBD-Workflows.
    \item Sie sollen zwei unterschiedliche Features implementieren. Entwickeln Sie beide Features parallel auf separaten Branches. Verfahren Sie entsprechend des TBD-Workflows.
    \item Stellen Sie sich vor, der Trunk ist bereit für einen weiteren Release (1.1).
    Setzen Sie eine Versionsnummer (1.1) und verfahren Sie entsprechend des TBD-Workflows.
    \item Sie erhalten ein Ticket für ein Refactoring einer bestehenden Funktion. Verfahren Sie entsprechend des TBD-Workflows. Stellen Sie sicher, dass alle Tests weiterhin erfolgreich durchgeführt werden.
    \item Stellen Sie sich vor, ein kritischer Fehler wurde entdeckt.
    Erstellen Sie einen kurzlebigen Branch, um den Fehler zu beheben. Verfahren Sie entsprechend des TBD-Workflows.
\end{enumerate}
